\section{Search Algorithms Used in CrowdHTN}

\subsection{Random DFS}

\subsection{Random BFS}
- trivially complete, will always encounter everything
- reference statistics (branching factor from Crowd, minimal plan depth from Lilotane) to show that 

\subsection{GBFS}
- inspired by PandaGBFS
- GBFS:
	- use a heuristic to grade each node
	- next node: best of all neighbors of the current node
	- then do a DFS on this
- GBFS is not complete
- might get stuck in an infinite loop
\todo{Quote to describe GBFS?}

The heuristic:
- determine the minimum number of actions/ reductions needed to solve the task network
- ignore any and all parameters and world state to simplify the computation
\todo{ensure the code actually does this!}
- iterative approach:
	- initial:
		- actions have remaining depth 0
		- reductions without children/ where all children are noops have depth 1
	- iterating:
		- reductions: the sum of the minimum depths of all subtasks + 1
		- tasks: the minimum over all reductions
- i.e., we heuristically try to find nodes with the minimum amount of work remaining
- each iteration gives us the remaining depth of at least 1 compound task, else we terminate
- i.e., efficient to compute
- tasks that do not receive a value are unresolvable and can be pruned

\begin{algorithm}
	\caption{GBFS heuristic calculation}
	task depths $\gets \{(t_c, 0) | t_c \in \text{ concrete tasks}\}$\;
	remaining tasks $\gets \{t_c | t_c \in \text{ compound tasks}\}$\;
	\While{task depths changed}{
		% get reduction depths
		\For{$t_c \in$ compound tasks}{
			reduction depths $= \emptyset$\;
			\For{$r \in$ reductions for $t_c$}{
				\If{depths of all subtasks are known}{
					reduction depths $\cup= \{ \max \{d | d \text{ is depth of a subtask of } r \} \}$\;
				}
			}
			
			\If{reduction depths $\neq \emptyset$}{
				task depths $\cup= \{ (t_c, \min \text{ reduction depths }) \}$\;
				remaining tasks $= \setminus \{ t_c \}$
			}
		}
	}
\end{algorithm}


\subsection{A-Star}
- same heuristic as GBFS
- also track the number of applied reductions so far
- gives us completeness (even without loop detection!) as the length of the path so far gives us a kind of BFS characteristic to the whole thing
- for each node we know exactly that any path
- we modify A* to terminate as soon as we find a goal for the first time
- could turn into an anytime algorithm by keeping up the search until optimal plan
