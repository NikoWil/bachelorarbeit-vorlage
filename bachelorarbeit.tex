% Vorlage für eine Bachelorarbeit - 2012-2013 Timo Bingmann

% Dies ist nur eine Vorlage. Strikte Vorgaben wie die Bachelorarbeit auszusehen
% hat gibt es nicht. Darum können auch alle Teile angepasst werden.

% TODO: change away from the enabledeprecatedfontcommands
\documentclass[enabledeprecatedfontcommands,12pt,a4paper,twoside]{scrartcl}

% Diese (und weitere) Eingabedateien sind in UTF-8
\usepackage[utf8]{inputenc}

% Verwende gute Type 1 Font: Latin Modern
\usepackage[T1]{fontenc}
\usepackage{lmodern}

% Sprache des Dokuments (für Silbentrennung und mehr)
\usepackage[german,english]{babel}

% Seitengröße - verwende fast die ganze A4 Seite
\usepackage[tmargin=22mm,bmargin=22mm,lmargin=20mm,rmargin=20mm]{geometry}

% Einrückung und Abstand zwischen Paragraphen
\setlength\parskip{\smallskipamount}
\setlength\parindent{0pt}

% Einige Standard-Mathematik Pakete
\usepackage{latexsym,amsmath,amssymb,mathtools,textcomp}

% Unterstützung für Sätze und Definitionen
\usepackage{amsthm}

\newtheorem{Satz}{Satz}[section]
\newtheorem{Definition}[Satz]{Definition}
\newtheorem{Lemma}[Satz]{Lemma}

\numberwithin{equation}{section}

% Deutsches Literaturverzeichnis
\usepackage{bibgerm}

% Unterstützung zum Einbinden von Graphiken
\usepackage{graphicx}

% Pakete die tabular und array verbessern
\usepackage{array,multirow}

% Kleiner enumerate und itemize Umgebungen
\usepackage{enumitem}

\setlist[enumerate]{topsep=0pt}
\setlist[itemize]{topsep=0pt}
\setlist[description]{font=\normalfont,topsep=0pt}

\setlist[enumerate,1]{label=(\roman*)}

% TikZ für Graphiken in LaTeX
\usepackage{tikz}
\usetikzlibrary{calc}

% Aktuelle Section und Untersection am Seitenkopf
\usepackage{fancyhdr}

\fancypagestyle{plain}{
  \fancyhead{}
  \fancyfoot{}
  \fancyfoot[LE,RO]{\normalsize\thepage}
  \renewcommand{\headrulewidth}{0pt}
  \renewcommand{\footrulewidth}{0pt}
}

\fancypagestyle{normal}{
  \setlength{\headheight}{20pt}
  \setlength\footskip{32pt}
  \fancyhead{}
  \fancyhead[LE]{\normalsize\textsc{\nouppercase{\leftmark}}}
  \fancyhead[RO]{\normalsize\textsc{\nouppercase{\rightmark}}}
  \fancyfoot{}
  \fancyfoot[LE,RO]{\normalsize\thepage}
  \renewcommand{\headrulewidth}{0.4pt}
  \renewcommand{\footrulewidth}{0pt}
}

% Hyperref für Hyperlink und Sprungtexte
\usepackage{xcolor,hyperref}

\hypersetup{
  pdftitle={Hier den Titel der Arbeit},
  pdfauthor={Hier den Author der Arbeit},
  pdfsubject={Stichworte, weiteres Stichwort},
  colorlinks=true,
  pdfborder={0 0 0},
  bookmarksopen=true,
  bookmarksopenlevel=1,
  bookmarksnumbered=true,
  linkcolor=blue!60!black,
  %linkcolor=black,
  citecolor=blue!60!black,
  urlcolor=blue!60!black,
  filecolor=green!60!black,
  pdfpagemode=UseNone,
  unicode=true,
}

% Paket zum Setzen von Algorithmen in Pseudocode mit kleinen Stilanpassungen
\usepackage[ruled,vlined,linesnumbered,norelsize]{algorithm2e}
\DontPrintSemicolon
\def\NlSty#1{\textnormal{\fontsize{8}{10}\selectfont{}#1}}
\SetKwSty{texttt}
\SetCommentSty{emph}
\def\listalgorithmcfname{Algorithmenverzeichnis}
\def\algorithmautorefname{Algorithmus}
\let\chapter=\section % repariert ein Problem mit algorithm2e

\DeclareOldFontCommand{\bf}{\normalfont\bfseries}{\mathbf}

\usepackage{todonotes}
\usepackage{comment}
\usepackage{algorithm2e}

\newtheorem{definition}{Definition}

\begin{document}

%%%%%%%%%%%%%%%%%%%%%%%%%%%%%%%%%%%%%%%%%%%%%%%%%%%%%%%%%%%%%%%%%%%%%%

\pagestyle{empty} % keine Seitenzahlen

% Titelblatt der Arbeit
\begin{titlepage}

  \begin{center}\large

    \quad\includegraphics[height=17mm]{kit_logo_de.pdf} \hfill
    \includegraphics[height=20mm]{grouplogo-algo-blue.pdf}\quad\null

    \vfill

    Bachelorarbeit
    \vspace*{2cm}

    {\bf\huge Malleable TOHTN Planning \\using CrowdHTN and Mallob \par}
    % Siehe auch oben die Felder pdftitle={}
    % mit \par am Ende stimmt der Zeilenabstand

    \vfill

    Niko Wilhelm

    \vspace*{15mm}

    Abgabedatum: 19.10.2012

    \vspace*{45mm}

    \begin{tabular}{rl}
      Betreuer: & Prof. Dr. Peter Sanders \\
      & M.Sc. Dominik Schreiber \\
    \end{tabular}
    
    \vspace*{10mm}

    Institut für Theoretische Informatik, Algorithmik \\
    Fakultät für Informatik \\
    Karlsruher Institut für Technologie

    % English:
    % Institute of Theoretical Informatics, Algorithmics \\
    % Department of Informatics \\
    % Karlsruhe Institute of Technology

    \vspace*{12mm}
  \end{center}

\end{titlepage}

%%%%%%%%%%%%%%%%%%%%%%%%%%%%%%%%%%%%%%%%%%%%%%%%%%%%%%%%%%%%%%%%%%%%%%

\vspace*{0pt}\vfill

\hrule\medskip

Hiermit versichere ich, dass ich diese Arbeit selbständig verfasst und keine anderen, als die angegebenen Quellen und Hilfsmittel benutzt, die wörtlich oder inhaltlich übernommenen Stellen als solche kenntlich gemacht und die Satzung des Karlsruher Instituts für Technologie zur Sicherung guter wissenschaftlicher Praxis in der jeweils gültigen Fassung beachtet habe.

\bigskip

\noindent
Karlsruhe, den \today

% Unterschrift (handgeschrieben)

\vspace*{5cm}

\clearpage

%%%%%%%%%%%%%%%%%%%%%%%%%%%%%%%%%%%%%%%%%%%%%%%%%%%%%%%%%%%%%%%%%%%%%%

\vspace*{0pt}\vfill

\selectlanguage{german}
\begin{abstract}
\centerline{\bf Zusammenfassung}

Hier die deutsche Zusammenfassung.

Ich bin Blindtext. Von Geburt an. Es hat lange gedauert, bis ich begriffen habe, was es bedeutet, ein blinder Text zu sein: Man macht keinen Sinn. Man wirkt hier und da aus dem Zusammenhang gerissen. Oft wird man gar nicht erst gelesen. Aber bin ich deshalb ein schlechter Text? Ich weiß, dass ich nie die Chance haben werde im Stern zu erscheinen. Aber bin ich darum weniger wichtig? Ich bin blind! Aber ich bin gerne Text. Und sollten Sie mich jetzt tatsächlich zu Ende lesen, dann habe ich etwas geschafft, was den meisten ,,normalen`` Texten nicht gelingt.

Ich bin Blindtext. Von Geburt an. Es hat lange gedauert, bis ich begriffen habe, was es bedeutet, ein blinder Text zu sein: Man macht keinen Sinn. Man wirkt hier und da aus dem Zusammenhang gerissen. Oft wird man gar nicht erst gelesen. Aber bin ich deshalb ein schlechter Text? Ich weiß, dass ich nie die Chance haben werde im Stern zu erscheinen. Aber bin ich darum weniger wichtig? Ich bin blind! Aber ich bin gerne Text.

\end{abstract}

\vfill

\selectlanguage{english}
\begin{abstract}
\centerline{\bf Abstract}

And here an English translation of the German abstract.

I'm blind text. From birth. It took a long time until I realized what it means to be random text: You make no sense. You stand here and there out of context. Frequently, they do not even read. But I have a bad copy? I know that I will never have the chance of appearing in the. But I'm any less important? I'm blind! But I like to text. And you should see me now actually over, then I have accomplished something that is not possible in most ``normal'' copies.

I'm blind text. From birth. It took a long time until I realized what it means to be random text: You make no sense. You stand here and there out of context. Frequently, they do not even read. But I have a bad copy? I know that I will never have the chance of appearing in the. But I'm any less important? I'm blind! But I like to text.

\end{abstract}
\selectlanguage{german}

\vfill\vfill\vfill
\clearpage

%%%%%%%%%%%%%%%%%%%%%%%%%%%%%%%%%%%%%%%%%%%%%%%%%%%%%%%%%%%%%%%%%%%%%%

\vspace*{0pt}\vfill

\section*{Danksagungen}

Thanks to i10pc135 which suffered much to make the experimental evaluation possible.

\vfill\vfill\vfill
\clearpage

%%%%%%%%%%%%%%%%%%%%%%%%%%%%%%%%%%%%%%%%%%%%%%%%%%%%%%%%%%%%%%%%%%%%%%

\pagestyle{normal}
% markiere sections im Seitenkopf links und subsections rechts
\renewcommand\sectionmark[1]{\markboth{\thesection\quad\MakeUppercase{#1}}{\thesection\quad\MakeUppercase{#1}}}
\renewcommand\subsectionmark[1]{\markright{\thesubsection\quad\MakeUppercase{#1}}}

% Inhaltsverzeichnis
\tableofcontents

\clearpage

%%%%%%%%%%%%%%%%%%%%%%%%%%%%%%%%%%%%%%%%%%%%%%%%%%%%%%%%%%%%%%%%%%%%%%

\listoffigures
\listoftables
\listofalgorithms

\clearpage

%%%%%%%%%%%%%%%%%%%%%%%%%%%%%%%%%%%%%%%%%%%%%%%%%%%%%%%%%%%%%%%%%%%%%%

\section{Introduction}
\subsection{Motivation}
\subsection{Research Goal}
- Provide a performant parallel TOHTN planner by improving upon the Crowd planner
- Provide integration of TOHTN into the Mallob malleable load balancer
- Compare performance of parallel to malleable TOHTN planning
- 
\subsection{Thesis Overview}

\section{Preliminaries}

\ref{prelim: tohtn problems}

\subsection{Planner Properties}
- completeness: we always find a plan if it exists
- correctness: kinda obvious
- systematicity (\cite{holler2020htn}): for search we explore each search node at most once
- optimality: shortest plan (few planners, Lilotane maybe?)

\subsection{TOHTN Formalism}

\subsubsection{Defining TOHTN Planning Problems}
Both HTN and TOHTN planning are based on the idea of decomposing a list of initial tasks down into smaller subtasks until those subtasks can be achieved by simple actions.

\todo{quote more formalisms}
Multiple definitions for HTN planning exist. In this work we build on the definition introduced in \cite{georgievski2015htn}.

\begin{definition} % predicate
	A \textbf{predicate} consists of two parts. Firstly a predicate symbol $p \in \mathcal{P}$ where $\mathcal{P}$ is the finite set of predicate symbols. Secondly of a list of terms $\tau_1, \ldots, \tau_k$ where each term $\tau_i$ is either a constant symbol $c \in \mathcal{C}$, with $\mathcal{C}$ being the finite set of constant symbols, or a variable symbol $v \in \mathcal{V}$, where $\mathcal{V}$ is the infinite set of variable symbols. \\
	The set of all predicates is called $\mathcal{Q}$.
\end{definition}
With the definition of a predicate in place, we can then define a grounding as well as our world state.
\begin{definition} % grounding
	A \textbf{ground predicate} is a predicate where the terms contain no variable symbols or, in other words, a predicate that contains only constant symbols.
\end{definition}
\begin{definition} % state
	A \textbf{state} $s \in 2^{\mathcal{Q}}$ is a set of ground predicates for which we make the closed-world-assumption. Under the closed-world-assumption, only positive predicates are explicitly represented in $s$. All predicates not in $s$ are implicitly negative.
\end{definition}


\begin{definition} % primitive task/ action
	With $T_p$ the set of primitive task symbols, a \textbf{primitive task} $t_p$ is defined as a triple $t_p(\Tilde{t}_p (a_1, \ldots, a_k), pre(t_p), eff(t_p))$. $\Tilde{p} \in T_p$ is the task symbol, $a_1, \ldots, a_k \in \mathcal{C} \cup \mathcal{V}$ are the task arguments, $pre(t_p) \in 2^{\mathcal{P}}$ the preconditions and $eff(t_p) \in 2^{\mathcal{P}}$ the effects of the primitive task $t_p$. We further define the positive and negative preconditions of $t_p$ as $pre^+(t_p) := \{p \in pre(t_p) : p \text{ is positive}\}$ and $pre^-(t_p) := \{p \in pre(t_p) : p \text{ is negative}\}$. We define $eff^+(t_p)$ and $eff^-(t_p)$ analogously. \\
	We call a fully ground primitive task an \textbf{action}.
\end{definition}

As preconditions and effects may not be concerned with the whole world state the closed-world assumption does not apply to them. To any HTN instance we could create an equivalent one where each precondition and effect cares about the whole world state. This would be achieved by instantiating all the "don't care" terms in preconditions and effects with all possible combinations of predicates. Doing this would, however, come at the price of a huge blowup of our planning problem. \\

\begin{definition} % applicable, application
	An action $t_p$ is \textbf{applicable}
	- $pre^+$ is fully contained in the state $s$
	- none of $pre^-$ are contained in $s$
	- the \textbf{application} of a task gives us a new state $s' = (s \setminus eff^-(t_p)) \cup eff^+(t_p)$
\end{definition}

\begin{definition} % compound task/ abstract task
	We define a \textbf{compound task} as $t_c = \Tilde{t}_c(a_1, \ldots, a_k)$, where $\Tilde{t_c} \in T_c$ is the task symbol from the finite set of compound task symbols $T_c$ and $a_1, \ldots, a_k$ are the task arguments.
\end{definition}
Primitive and compound tasks together form task networks. In places where both can be used, we will refer to them simply as tasks $t \in T$.

\begin{definition} % task network
	Let $T = T_p \bigcup T_c$ be a set of primitive and compound tasks. A task network is a tuple $\tau = (T, \psi)$ consisting of tasks $T$ and constraints $\psi$ between those tasks.
\end{definition}

\begin{definition} % method, reduction
	Let $M$ be a finite set of method symbols and $T = T_p \bigcup T_c$ a set of primitive and compound tasks. A \textbf{method} $m = (\Tilde{m}(a_1, \ldots, a_k), t_c, pre(m), subtasks(m), constraints(m))$ is a tuple consisting of the method symbol $\Tilde{m}$, the method arguments $a_1, \ldots, a_k$, the associated compound task $t_c \in T_c$ the method refers to, a set of preconditions $pre(m) \in 2^{\mathcal{P}}$, a set of tasks $subtasks(m) = \{t_1, \ldots, t_l\}, t_i \in T$ and a set of constraints $c_1, \ldots, c_m$ defining relationships between the tasks in $subtasks(m)$. Any arguments appearing in $t_c, pre(m), subtasks(m)$ must also appear in $a_1, \ldots, a_k$.\\
	In TOHTN planning, $constraints(m)$ is implicitly set s.t. the subtasks $t_1, \ldots, t_l$ are totally ordered. \\
	We call a fully ground method a \textbf{reduction}.
\end{definition}
\todo{link directly to resolution of a task network if resolving is defined before method!}
Each method $m$ has exactly one associated compound task $t_c$. However, multiple methods $m_1, \ldots, m_k$ may be associated with a single compound task $t_c$. Additionally, while any arguments of $t_c$ must be present in $m$, the contrary is not true and $m$ may have arguments not present in $t_c$, i.e., $m$ is not fully determined by $t_c$. As a result methods present choice points both in the choice of method itself as well as through the argument instantiation. \\


\begin{definition}
	Test
\end{definition}

\
\subsection{Techniques to solve TOHTN planning problems}

\subsection{SAT-based}
- SAT-based
- often with a BFS-like characteristic (Tree-Rex, Lilotane, TotSat)

\subsubsection{Progression Search}
\cite{holler2020htn}
	- progression search is one of the best known search algorithms
	- generate plans in a forward way
	- always resolve a task that has no more open predecessors with the ordering constraints
		- own: for TOHTN: we always have exactly 1 task we want to process next!
	- progression search planners always know the current (world) state, can use this information for heuristics, pruning etc
	
	- other planners search partial plans but not in order, they thus cannot know the current world state
	
\begin{algorithm}
	\caption{Classical Progression Search as introduced in \cite{holler2020htn}}
	$fringe \gets \{ (s_0, tn_I, \epsilon)\}$\;
	\While{$fringe \neq \emptyset$}{
		$n \gets fringe.pop()$\;
		\If{$n.isgoal$}{
			\textbf{return} $n$\;
		}
		$U \gets n.unconstrainedNodes$\;
		\For{$t \in U$}{
			\eIf{$isPrimitive(t)$}{
				\If{$isApplicable(t)$}{
					$n' \gets n.apply(t)$\;
					$fringe.add(n')$\;
				}
			}{
				\For{$m \in t.methods$}{
					$n' \gets n.decompose(t, m)$\;
					$fringe.add(n')$\;
				}
			}
		}
	}
\end{algorithm}

- search-based
- lifted vs grounded
	- Lilotane, HyperTensioN
	- Panda, CrowdHTN, Tree-Rex
	- lifted: more general, less pruning?

\subsection{Malleable Algorithms}

\section{The CrowdHTN Planner}
\textbf{Cooperative randomized workstealing for Distributed HTN}


\section{TOHTN Metadata}
- researchers in TOHTN planning have collected a number of test instances for TOHTN planning
- provide some analysis of those instances in the context of modelling TOHTN planning as graph search
- provide a foundation to discuss the effects of changes and improvements in the crowd planning framework

\section{Implementation}

\section{Improvements to CrowdHTN}

\subsection{Reducing Memory Consumption}

\subsubsection{Efficiently Storing the Preceding Plan}
\begin{itemize}
	\item Use the ImmutableStack<T> structure
	\item 
\end{itemize}

\subsubsection{Reducing Copies of the World State}

\subsubsection{Only Saving 'Potentially Interesting' Nodes}

\subsection{Making Crowd More CPU Efficient}

\subsection{Efficiently Hashing Nodes of the Search Graph}
\todo{Time is not actually constant for world state -> what is the expected time?}
Overview:
\begin{itemize}
	\item the hash of a node consists of two parts
	\item 1: the hash of the open tasks
	\item 2: the hash of the world state
	\item We do not care about the hash of the preceding plan. Nodes with open tasks and world state equivalent have equivalent plans leading to a goal (somewhat similar to the Nerode relation) and we do not care about optimality. How he reach this point with equivalent remaining options is thus not of interest to us
	\item The hash of the world state can be large, but the number of elements is ultimately bound for any particular instance
	\item The length of the preceding plan is unbounded
\end{itemize}
Open tasks:
\begin{itemize}
	\item Care needs to be taken to use an order independent hash function for the world state, as the underlying representation as a set does not guarantee us any particular iteration order, especially between nodes (might depend on the order of things inserted into the set!)
	\item Alternatively we could have chosen some other ordered representation, e.g. maintain the world state as a sorted list of predicates with a defined order. This would imply extra work we are unwilling to do.
	\item For open tasks, we save not only the task itself but additionally the hash of all open tasks from first to current one
	\item The order in which we hash open tasks is from oldest to newest one
	\item On applying a Reduction, we push the new open tasks onto the open task stack and compute each of their hashes combined with their predecessors. Each of these hashing operations is completed in O(1)
	\item Effectively, this means that each task is hashed exactly once
	\item As we already have to push each task onto the open tasks, inserting an additional O(1) operation on pushing does not change the asymptotic runtime
\end{itemize}
World state:
\todo{this is not implemented yet, take care to actually bring it into Crowd}
\begin{itemize}
	\item In section XXX \todo{add section, refer to it here} we discussed how each copy of the world state can be shared by many search nodes to reduce the memory footprint
	\item Instead of hashing the world state each time on demand, we can store a shared tuple of world state and it's hash
	\item This way, we only hash the world state once, reusing the computation
	\item Is this actually a time saving?
	\todo{Only put node into loop detection as it is explored, so we can always keep exploring the local node (the node we are colliding with may be arbitrarily far back in our fringe)}
	\todo{Evaluate how many world state hashes we actually compute in both cases. Hashing a world state of a node which is ultimately not explored (and which is otherwise not needed) is wasted.
	Number of computations on lazy hashing: number of nodes created - number of nodes remaining in fringe on plan found}
	\item Future work: the hash of the world state is order independent (sum of squares of individual hashes), to not worry about forcing any fixed iteration order. Utilize this and wrapping maths to hash only the differential of 
\end{itemize}

\subsection{Preceding Plan}
\todo{Section no longer valid for current crowd}
In the initial implementation each node stored the full preceding plan as a sequence of all reductions that were applied so far. This leads to roughly quadratic overhead (sum over 1..n, only roughly as not each step increases the length. Wait, is it roughly, then? Probably, as the fraction of steps that increment the preceding plan should be kinda constant)
This duplication was not needed. The newer implementation instead stores an optional<reduction> in each node. I.e., the preceding reduction is stored if one exists, nothing if there isn't one. When the preceding plan is needed, either for communication or to extract a plan, the current search path is iterated and all reductions are accumulated.

\subsection{Lazy Instantiation of Child Nodes}
lazy instantiation works on the basis of finding all free variables of a method and creating Reductions based on all possible combinations

Initial implementation: instantiate all possible reductions, filter out any with not fulfilled preconditions,then shuffle them

Problems: this spends both time and memory instantiating reductions that might never be needed for the rest of the search
We effectively save not only the current path, but also follow all possible branches to a distance of 1

Solution: lazily create reductions as needed
How to do this (first way):
adapt the argiterator from Lilotane into the CrowdHTN code base. Adapt it to only substitute the arguments that are not already determined by the corresponding task (arguments)

To achieve randomization:
each domain is iterated to create the substitutions
Each time we build such an iterator, we randomize the order within the domains for this specific iterator
This will lead to different orders

Further ideas:
each time domains 1..k have been fully iterated, increment k+1 by 1, then shuffle the order of domains 1..k

For n total domains, each time domains 1..n-1 have been iterated, remove the current value from domain n, then shuffle all domain orders

Compare the runtimes of eager and lazy instantiation, check at which point it is worth it to incur the (potential) additional overhead of lazy instantiation
Compare on multiple domains?
Check different metrics for comparison (size of domains, number of parameters, number of potential children (product of sizes))

Potential problems:
We need quite a bit of state (domains, current index into each domain) to perform lazy instantiation
The order is not truly random. We iterate some domains faster/ more often than others. What if the important change is in a domain which is iterated slowly? More random order makes this easier

A potential solution: space filling curves
Advantages: little state (can just be incremented), iterates all dimensions equally
Disadvantage: fixed order. With shuffling within each domain might be random again

Space filling curves come with the restriction of being strictly continuous
We do not need this property. All we are interested in is an easy to compute fixed order in which the whole space is iterated where each permutation is hit exactly once
We want only self-avoiding curves, to not hit any instantiation twice (could loop detection just fix that? But it would be a bad fix)

\subsection{Using Domain Meta-Information}
inspiration taken from HyperTensioN

taking preconditions of tasks and lifting them up into methods
Only do this if the precondition is guaranteed to also apply to the method (cannot be achieved before the respective task?)
This allows us to stop exploring paths earlier

\subsection{Global and Memory Efficient Loop Detection}
So far: each worker has a hash set of each node ever encountered (note: we define node equality through the sequence of open tasks and the set of predicates that is the world state. Depth in the graph and preceding plan are ignored)
Advantages: allows for perfect local loop detection, as we can always fall back to equality checks in case of hash collisions.
Problems: This approach leads to a massive memory overhead, as the world state is duplicated countless times. This also leads to increased run times just to manage the growing hash set.
This is also not suited for global loop detection. Global loop detection would need some mechanism to communicate seen nodes. Nodes are too large to communicate all of them, though. If we only communicate some nodes instead of all we run the risk of a node "going past" that barrier of known nodes and still exploring the swathes of known nodes beyond the "barrier"
TODO: what about open tasks (memory consumption)? Or was that an ImmutableStack with little overhead?

Idea: drop the precondition of perfect loop detection with no false positives.
For loop detection use a bloom filter with a set of hash functions for search nodes.
This comes with a fixed, configurable memory overhead per worker. As a result, overall memory consumption should
- drop
- be more predictable
To communicate nodes we can just communicate the bit array that makes up our filter and combine them via bitwise OR. This ensures that communication volume is also fixed. It is independent of both size of nodes (i.e. length of open tasks sequence) and number of explored nodes (since last round of communication).
In addition to communicating the bloom filter we communicate the number of newly added nodes. Then each worker knows an upper bound for the total number of nodes that are part of it's bloom filter (might be an overestimation if two workers insert identical nodes as the node would be counted twice). As bloom filter performance/ false positives degrade with the number of elements inside it, we can restart our work depending on the number of elements in the bloom filter.

- loop detection information is shared at most every second (or however long it takes for the communication to complete)
- as a result, when the root node increases it's version, a broadcast is started that communicates the new version to everyone else
- the version in normal messages is still kept. In case a message arrives before the version broadcast, the version is increased even earlier. New nodes likely get their version with the first work package.

\subsubsection{Restarts Under Loop Detection}
- bloom filters are of fixed size. If we propose a maximum false-positive rate, they will fill up at some point
- this necessitates the need for restarts
- bloom filter architecture: for local loop detection, use an expanding bloom filter (citation!)
- this allows us to not incur any restarts due to local state
- for global loop detection, we use a fixed-size bloom filter
- if a node is encountered twice for local loop detection, add it to the global loop detector, to communicate

- we know that in mallob, the tree of workers is always a 'full' tree, except for the lowest level
- i.e. we may be missing some nodes to the right end of the lowest level
- especially, the root node always survives
- as a result, the root node always receives all the messages about shared loop detection data
- i.e., if the global loop detection bloom filter on any node is full, it follows that the global loop detection bloom filter on the root node is also full
- the opposite does not hold (i.e. a new node enters just as the root node bloom filter fills up)
- as a result, we can simply delegate the question of whether to restart to the root node and have it done in a centralized way, simplifying our communication patterns

\subsubsection{Reaching Consensus on Search Version}
- each search version corresponds to a restart with subsequent doubling in size of the global loop detector
- keeping versions in sync is of importance
- we do not want to loose any work
- we do not want to insert search nodes from a wrong version into our bloom filter (especially old node into new filter, other way around is discarded either way)
- each message between workers is extended by their internal version
- if the version of a received message is higher than the internal version, the internal version is increased and then acted accordingly
- new workers get updated to the current version the first time they ask for a work package
- work requests and responses do not suffice as a version updating mechanism, as work packages may be arbitrarily large and messages arbitrarily rare
- 

\subsubsection{Completeness Under Loop Detection}
- any single iteration may be incomplete, e.g. if initial node and goal node hash to the same values
- with restarts, we have uniform hashes, compute the chance of colliding each time
- this necessitates changing seeds with each restart
- soon, we will not collide
- yay, plan can be found!

- we do loose termination
- so far, termination could be known if the search tree gets too deep (cite limits known from other TOHTN paper)
- however, this is infeasible, as RAM is actually finite (refer to known branching factors!)
- i.e., we do not know the hash of a goal node, as a result we cannot exclude the possibility of it simply being cut off each time so far
- we also do not know the hash of all nodes just before a goal node, etc (the path could always only be a single node wide)
- as a result, we cannot terminate at any point, as we cannot exclude the plan hiding in some always-cut-off part of the search space
- loosing termination is not a big deal anyways, I guess?



\section{Theoretical Improvements to the Crowd Planner}
\section{Search Algorithms Used in CrowdHTN}

\subsection{Random DFS}

\subsection{Random BFS}
- trivially complete, will always encounter everything
- reference statistics (branching factor from Crowd, minimal plan depth from Lilotane) to show that 

\subsection{GBFS}
- inspired by PandaGBFS
- GBFS:
	- use a heuristic to grade each node
	- next node: best of all neighbors of the current node
	- then do a DFS on this
- GBFS is not complete
- might get stuck in an infinite loop
\todo{Quote to describe GBFS?}

The heuristic:
- determine the minimum number of actions/ reductions needed to solve the task network
- ignore any and all parameters and world state to simplify the computation
\todo{ensure the code actually does this!}
- iterative approach:
	- initial:
		- actions have remaining depth 0
		- reductions without children/ where all children are noops have depth 1
	- iterating:
		- reductions: the sum of the minimum depths of all subtasks + 1
		- tasks: the minimum over all reductions
- i.e., we heuristically try to find nodes with the minimum amount of work remaining
- each iteration gives us the remaining depth of at least 1 compound task, else we terminate
- i.e., efficient to compute
- tasks that do not receive a value are unresolvable and can be pruned

\begin{algorithm}
	\caption{GBFS heuristic calculation}
	task depths $\gets \{(t_c, 0) | t_c \in \text{ concrete tasks}\}$\;
	remaining tasks $\gets \{t_c | t_c \in \text{ compound tasks}\}$\;
	\While{task depths changed}{
		% get reduction depths
		\For{$t_c \in$ compound tasks}{
			reduction depths $= \emptyset$\;
			\For{$r \in$ reductions for $t_c$}{
				\If{depths of all subtasks are known}{
					reduction depths $\cup= \{ \max \{d | d \text{ is depth of a subtask of } r \} \}$\;
				}
			}
			
			\If{reduction depths $\neq \emptyset$}{
				task depths $\cup= \{ (t_c, \min \text{ reduction depths }) \}$\;
				remaining tasks $= \setminus \{ t_c \}$
			}
		}
	}
\end{algorithm}


\subsection{A-Star}
- same heuristic as GBFS
- also track the number of applied reductions so far
- gives us completeness (even without loop detection!) as the length of the path so far gives us a kind of BFS characteristic to the whole thing
- for each node we know exactly that any path
- we modify A* to terminate as soon as we find a goal for the first time
- could turn into an anytime algorithm by keeping up the search until optimal plan


\subsection{Loop Detection}
\todo{Make sure to always talk about isomorphic task networks instead of equivalent? Quote \cite{holler2021loop}. Define isomorphism!}
\todo{quote about loop detection in graph search?}
\todo{quote about distributed loop detection in graph search?}
This section will discuss loop detection as it is used in (TO)HTN planning in general. It will start with an overview over loop detection in other HTN planners in section \ref{ld - history others}. This is followed by a discussion of the simplifying assumptions we can make for TOHTN planning (section \ref)
- distributed loop detection (communication and merge operations become important!)
- perfect loop detection
- probabilistic loop detection (approximate membership query)

- \cite{magnaguagno2020hypertension} domains can have introduce infinite loops, without some kind of loop detection we loose completeness (and performance)


\subsubsection{Loop Detection in Other HTN Planners}
\label{ld - history others}
Loop detection in HTN planning is a recent phenomenon and was introduced in 2020 by the HyperTensioN planner (\cite{magnaguagno2020hypertension}) with the so-called 'Dejavu' technique. Dejavu works by extending the planning problem, introducing primitive tasks and predicates that track and identify when a particular recursive compound task is being decomposed. These new primitive tasks are invisible to the user. Information about recursive tasks is stored externally to the search as to not loose it during backtracking. Dejavu comes with performance advantages and protects against infinite loops. However, as Dejavu only concerns itself with information about the task network but ignores the world state it may have false positives. This was also noted by \cite{holler2021loop} and means that HyperTensioN is not complete. \cite{holler2021loop} further nodes that the loop detection is limited in that it only finds loops in a single search path but cannot detect if multiple paths lead to equivalent states. \\
In response to \cite{magnaguagno2020hypertension}, loop detection was introduced to the PANDA planner in \cite{holler2021loop}. Similar to HyperTensioN, PANDA keeps it's loop detection information separate in a list of visited states, $\mathcal{V}$. Search nodes, identified by a tuple $(s, tn)$ of world state $s$ and task network $tn$, are only added to the fringe if they are not contained in $\mathcal{V}$. To speed up comparisons, $\mathcal{V}$ is separated into buckets according to a hash of $s$. To then identify whether $tn \in \mathcal{V}[s]$ multiple algorithms are proposed. In the sub-case of TOHTN planning, both an exact comparison of the sequence of open tasks as well as an order-independent hash of the open tasks, called \textit{taskhash} are used. Similar to HyperTensioN, using a hash to identify equal task networks can lead to false positives and an incomplete planner. Both hashing-based and direct comparison as used in PANDA have a performance cost linear in the size of $tn$. The loop detection in PANDA improves upon the one in HyperTensioN insofar as it is able to detect loops where equivalent states where reached independently.

\begin{comment}
- loop detection in HTN planning is relatively new, being introduced by \cite{magnaguagno2020hypertension} and \cite{holler2021loop}
- multiple other planners do perform loop detection
- HyperTensioN \cite{magnaguagno2020hypertension}
	- transform the domain, adding primitive tasks that mark when a particular, recursive, task is being decomposed
	- use an external cache to store this information (else it would be lost in backtracking)
	- can also fall back on comparing the full search state (needed, if no parameters are present in the recursive task, to identify specific instances?)
	- limited, as it does not consider the world state, only the decompositions!, planner no longer complete \cite{holler2021loop}
- PandaGBFS \cite{holler2021loop}
	- search nodes are stored in a fringe
	- separate visited list $\mathcal{V}$
	- only add nodes to the fringe if they are not in $\mathcal{V}$
	- a node is identified by a pair $(s, tn)$, $s$ state, $tn$ task network (as in crowd!)
	- $\mathcal{V}$ is separated in buckets identified by states, access bucket based on 64 bit integer hashing (of $s$?)
	- problems with task network isomorphism, as Panda deals with HTN instead of only TOHTN, ordering relations matter!
	- simplifying isomorphism -> accept false positives (but not false negatives!), may cost completeness
	- exact isomorphism for TOHTN (TN -> sequence of tasks, simple)
	- hash (for TOHTN) that discards ordering information (nicer for HTN), only tracks multiplicity of all open tasks
	- hash can be computed in linear time (as no ordering)
	- hash is used both for making search more efficient (less task network comparisons!) and as an overapproximation
\end{comment}

\subsubsection{Assumptions in Loop Detection for CrowdHTN}
\label{ld - tohtn simplifications}
To design the loop detection in CrowdHTN, we have both simplifying and complicating assumptions that we will discuss here. \\
While both PANDA and HyperTensioN are HTN planners, CrowdHTN concerns itself only with TOHTN planning. As a result, the remaining task network can be represented as a sequence of open tasks with the ordering constraints implicit in how the sequence is stored.
\todo{refer to the implementation part where the open tasks are represented as a sequence} \\
Crowd identifies search states as a tuple of $(s, tn)$ of world state $s$ and task network $s$, similar to PANDA and uses hashing to efficiently identify duplicate search states. As tasks of equivalent task networks are always in the same order, we can incorporate that order into our hash of $tn$ to reduce the number of collisions compared to PANDA's \textit{taskhash}. This will increase performance where we fall back to comparisons in case of collisions and reduce our false-positive rate in case we use probabilistic loop detection. 
\todo{Refer to the implementation and how we reduce hashing to O(1) time} \\
Both PANDA and HyperTensioN are sequential planners whereas CrowdHTN is highly parallel. This adds an additional design constraint to our loop detection. If we want to efficiently share information about visited states using the full state information becomes infeasible as full states would have to be communicated. If we perform loop detection only locally, we predict to suffer from decreased performance the higher the degree of parallelism. I.e., if two branches of our search tree contain the same search node, the chance that those two branches are encountered on different processors is higher the more processors we have.

\begin{comment}
	- keep the part where we store loop detection information externally
	- be sure we can always assume total order (and keep the ordering information to reduce false-positive rate!)
	- keep completeness of the planner
	- improve on the performance for hashing!
	- 
\end{comment}


\subsubsection{Perfect Loop Detection}
One simple way to perform loop detection which is similarly used in PANDA (\cite{holler2021loop}) is to use a hashset of visited states. The implementation in Crowd is slightly different from PANDA in that we use one combined hash for both world state and open tasks. Other than PANDA, CrowdHTN does incorporate the order of tasks into the hash, which should reduce collisions and makes the two levels of hashing less needed. \\
Using hashes combined with a full comparison provides us a perfect loop detection, i.e., neither false positives nor false negatives exist. This makes it a useful technique to benchmark other loop detection methods. However, both in the sequential and in the distributed case this technique suffers from performance problems. \\
In case of hash collisions, we have to fall back to a full comparison of world state $s$ and open tasks $tn$. While $s$ is bound in size by the total number of predicates, the size of $tn$ is effectively unbound \todo{quote about max. exponential depth!}. Additionally, we have to keep both $s$ and $tn$ around for all nodes ever encountered, increasing the memory footprint of our program.
In case of distributed loop detection, communicating full states would lead to a large communication overhead. Especially, we can expect each node to be larger than those sent as work packages, as those are optimized to have a small $tn$ which would not hold for nodes encountered in our loop detection.

\begin{comment}
	- Additionally, there is the full comparison for the world state
	- in practise this can be a problem, at the same time asymptotically it should not matter as it will be dwarfed by the 
	\todo{get some data on how many nodes share a world state on average as well as sizes of world states -> number of hash operations per world state!}
	- and the full comparison of open tasks
	- we cannot just free the open tasks and world state that are no longer needed! Both time and memory footprint are worse
	- worse memory and time complexity
	- however, it is a useful benchmark as to how many loops we *should* expect
\end{comment}

\subsubsection{Bloom Filters}
\todo{Write about why a quotient filter is not the solution! (or at least why it was not chosen)}
Quotient filter:
- do we need the original elements to re-insert them?
- do we need to communicate whole hashes to combine filters? (worse communication size!)


Advantages:
- communication takes less effort (a lot lower size of data!)
- hashing in O(1)? Or at least in a lot easier
- we can pre-hash the open tasks (can be of exponential depth (requires a good argument, as the path down the task network could be of width 1?) \todo{do the maths}), thus turning hashing into O(1)
- in case of hash collisions we need to walk the whole sequence of open tasks
- in practice this is even worse, as our open tasks are saved in a tree-like manner to allow sharing of the tasks between search nodes
- this means we wildly jump through memory for hashing, adding another constant factor
- we could save on this constant factor by duplicating the open tasks for each node and saving them sequentially, but leading to a lot higher memory footprint (might be quadratic compared to what it is already (if any fixed fraction of nodes have at least 2 children, maybe? \todo{some math}))
- compromise between performance and false-positive rate (better for correctness than just comparing a single hash)

\subsubsection{Distributed Loop Detection}
- two new considerations
	- memory footprint becomes more important for communication (effectively an all-to-all operation -> quote Sanders' book!)
	- efficient merge of loop detection data is needed
- communicating everything might still be inefficient for bloom filters ()

The goal of this section is to describe how we adapt it to be a malleable TOHTN planner by integrating it with Mallob, preserving both the completeness and scalability of CrowdHTN in the process. Before we get into the details, let's recall that CrowdHTN is already a moldable program according to the definition introduced in section \ref{prelim: malleability}, i.e., it may utilize any number of PEs as long as that number stays fixed during the run. We will now introduce a design that extends the parallel capabilities to achieve malleability. For this we need to address three main concerns.
\begin{itemize}
	\item Distributing the job information
	\item Integrating new PEs into a running job
	\item Dealing with PEs leaving the job while it runs
\end{itemize}
In the following sections we will address these problems in this order. Both distributing the job information and integrating new workers do not pose significant problems. Most time will be spent on the handling of disappearing PEs.
Due to the fact that we specifically integrate CrowdHTN with Mallob, in some parts we will have to refer to implementation details regarding how Mallob organizes the PEs assigned to a job as well as general message delivery.

\subsection{Distributing Jobs}
When a PE is assigned to a job, it needs to obtain a description of this job. In case of TOHTN planning, the choice is mostly between a lifted or ground TOHTN instance. Depending on pruning, a ground instance may be up to exponential in size \cite{behnke2020succinct}. Encoding and communicating such a ground instance would take up much time, which is why we decided to communicate our problem as a lifted instance. \\
With the lifted instance, we choose to simply take the textual hddl input (\cite{holler2020hddl}) and send it as-is. While this does occur the overhead of locally parsing the instance on each PE, communicating the parsed instance would involve re-encoding and effectively re-parsing it locally, too. \\
In malleable (TO)HTN planning there is a more general trade-off involved when it comes to precomputation. While parsing the instance is unavoidable, we can choose whether we want to spend time grounding and pruning our instance. It has been shown that grounding and pruning improve the planning performance (\cite{behnke2020succinct}) and allow for the computation of complex and good heuristics (\cite{holler2020htn}), but grounding, pruning and other precomputations are expensive operations themselves. As a result, a PE which is only assigned to our job for a short time may never perform any actual planning work before it is reassigned to the next job. For this reason, CrowdHTN takes an alternative path. The TOHTN instance is kept in lifted form. Instantiation is only performed as needed to explore the current search node. This allows CrowdHTN to start working immediately to utilize even short-lived PEs.
\begin{comment}
- grounded instance may be exponential in size
- exceedingly high communication cost
- lifted instance it is
- sending a parsed instance - we still need to encode and decode it for sending
- we would effectively have to build a parser for this
- we already do have a parser
- we simply encode the string and parse locally
- not a core issue to us, so we go for ease of implementation
\end{comment}

\subsection{Integrating New PEs Into Malleable CrowdHTN}
To integrate a new PE into a running TOHTN job, it needs both the general job description and part of the actual work to handle. In the previous section we explained how the job description is obtained, now we will focus on the work itself. \\
The efficient integration of new PEs into a running job is where work stealing shows it's strength. For work stealing, there is no functional difference between a PE which has locally run out of work and a new PE which has the job description but no work yet. Both will message other PEs at random to receive a new work package with no special handling required. As a result, a new PE can perform at full efficiency almost immediately, allowing our job to utilize resources as they become available.
\begin{comment}
- work stealing makes this easy
- there is functionally no difference between a new PE and a PE which has locally run out of work
- no special handling required at all

- only setup work: parsing, initiating heuristic
- CrowdHTN may be less efficient, but can work immediately, make use of very short-lived PEs
\end{comment}

\subsection{Handling PEs Leaving at Run Time}
The last challenge in designing a malleable CrowdHTN is the fact that PEs may disappear at any time. This represents a potential loss of information. The information loss presents itself in two ways. First, the loss of the local search fringe, if we do not communicate it to another PE and second, messages which may be lost in transit as their receiver no longer belongs to the same job. To deal with this, Mallob does allow us to detect locally when a PE is taken away from a job and additionally provides a message return mechanism. We will present our solutions to both cases with a focus on preserving the completeness property of CrowdHTN.

\subsubsection{Handling the Local Fringe}
When a local PE is unassigned from a job, we will loose the local search fringe. As Mallob signals a PE when it is unassigned from a job, we are however free to encode parts or all of the fringe and communicate them to another PE. This leaves us with a number of choices where we may trade-off data loss versus efficiency and communication. On this axis we discuss three choices
\begin{itemize}
	\item Encode and redistribute the whole local fringe
	\item Communicate the root of the local search space
	\item Communicate nothing, loose the local fringe
\end{itemize}

\paragraph{Encoding and redistributing the whole fringe}
Encoding and sending off the local fringe to another PE is, in a way, the easiest operation. No information is lost, preserving completeness in our planner. It does, however, come with a number of disadvantages. First, the local fringe may be arbitrarily large, especially considering that TOHTN planning is EXPSPACE-complete as seen in section \ref{prelim: tohtn complexity}. Encoding and communicating a large fringe is a very expensive operation which would increase the time from Mallob telling a PE to suspend itself until the PE actually is free for the next job. Second, receiving a large fringe would strain the memory of the receiving PE which may lead to deleting parts of it anyways to avoid crashes. Third, to avoid duplication of work as Mallob may reassign the PE to the old job, the local fringe would have to be cleared out. Doing so would weaken the effect of Mallob reassigning previously used PEs to the same job.
\begin{comment}
- the most complete operation
- nothing is lost
- nodes higher up in the tree of PEs may be more strained now (depending on the communication pattern)
- a very expensive operation
- take care to delete the local fringe to avoid duplication!
\end{comment}

\paragraph{Communicating the root of the local search space}
- easy and cheap
- duplication of work
	- re-exploring things
	- nodes that were sent off to other PEs
- how to deal with global loop detection
	- delete everything, too conservative, less performance
	- do not delete, cut off things wrongly
		- restarts deal with this increased false positive rate

- deal with disappearing workers
	- either design a scheme to preserve global knowledge or be able to deal with loss of information
	- in our case we deal with loss of information
	- two parts: loosing information stored in the fringe of a terminated node and loosing messages of dying workers
	- loosing information stored in the fringes:
		- multiple options:
			- send back nothing, loose parts of the search space
			- send back the root, redo parts of the search (also, loop detection!)
			- send back everything, takes much communication (also may duplicate the search space on resume)
			
			- our loop detection scheme already implies that we loose parts of the search space and we have measures in place to deal with this fact (restarts)
			- for this reason we go with this approach
			
\subsubsection{Handling Lost Messages}
	- loosing information due to lost messages:
		- Mallob has a mechanism to return messages
		- however, messages can still be lost (return message and original sender dies in the meantime)
		- we could change Mallob to send such messages to the root worker
		- however, this would turn the root into a bottleneck
		- again, we have mechanisms in place to deal with overall loss of information without loosing completeness
		- at the same time, we need to adapt our handling of return messages to ensure all workers stay in a valid state and do not get stuck
		- the worker may be replaced
		
	- getting wrong information (worker dies, is replaced, gets message meant for old worker)
	
	- we loose the ability to detect UNPLAN

\section{Grounding and Pruning}
\cite{holler2014language}
- a ground HTN instance has (worst-case) a size that is exponential in the arity of task names and predicates

\section{Experimental Evaluation}

\section{Future Work}

\clearpage

%%%%%%%%%%%%%%%%%%%%%%%%%%%%%%%%%%%%%%%%%%%%%%%%%%%%%%%%%%%%%%%%%%%%%%

\bibliographystyle{gerplain}
\bibliography{literatur}

\end{document}
