% Vorlage für eine Bachelorarbeit - 2012-2013 Timo Bingmann

% Dies ist nur eine Vorlage. Strikte Vorgaben wie die Bachelorarbeit auszusehen
% hat gibt es nicht. Darum können auch alle Teile angepasst werden.

% TODO: change away from the enabledeprecatedfontcommands
\documentclass[enabledeprecatedfontcommands,12pt,a4paper,twoside]{scrartcl}

% Diese (und weitere) Eingabedateien sind in UTF-8
\usepackage[utf8]{inputenc}

% Verwende gute Type 1 Font: Latin Modern
\usepackage[T1]{fontenc}
\usepackage{lmodern}

% Sprache des Dokuments (für Silbentrennung und mehr)
\usepackage[german,english]{babel}

% Seitengröße - verwende fast die ganze A4 Seite
\usepackage[tmargin=22mm,bmargin=22mm,lmargin=20mm,rmargin=20mm]{geometry}

% Einrückung und Abstand zwischen Paragraphen
\setlength\parskip{\smallskipamount}
\setlength\parindent{0pt}

% Einige Standard-Mathematik Pakete
\usepackage{latexsym,amsmath,amssymb,mathtools,textcomp}

% Unterstützung für Sätze und Definitionen
\usepackage{amsthm}

\newtheorem{Satz}{Satz}[section]
\newtheorem{Definition}[Satz]{Definition}
\newtheorem{Lemma}[Satz]{Lemma}

\numberwithin{equation}{section}

% Deutsches Literaturverzeichnis
\usepackage{bibgerm}

% Unterstützung zum Einbinden von Graphiken
\usepackage{graphicx}

% Pakete die tabular und array verbessern
\usepackage{array,multirow}

% Kleiner enumerate und itemize Umgebungen
\usepackage{enumitem}

\setlist[enumerate]{topsep=0pt}
\setlist[itemize]{topsep=0pt}
\setlist[description]{font=\normalfont,topsep=0pt}

\setlist[enumerate,1]{label=(\roman*)}

% TikZ für Graphiken in LaTeX
\usepackage{tikz}
\usetikzlibrary{calc}

% Aktuelle Section und Untersection am Seitenkopf
\usepackage{fancyhdr}

\fancypagestyle{plain}{
  \fancyhead{}
  \fancyfoot{}
  \fancyfoot[LE,RO]{\normalsize\thepage}
  \renewcommand{\headrulewidth}{0pt}
  \renewcommand{\footrulewidth}{0pt}
}

\fancypagestyle{normal}{
  \setlength{\headheight}{20pt}
  \setlength\footskip{32pt}
  \fancyhead{}
  \fancyhead[LE]{\normalsize\textsc{\nouppercase{\leftmark}}}
  \fancyhead[RO]{\normalsize\textsc{\nouppercase{\rightmark}}}
  \fancyfoot{}
  \fancyfoot[LE,RO]{\normalsize\thepage}
  \renewcommand{\headrulewidth}{0.4pt}
  \renewcommand{\footrulewidth}{0pt}
}

% Hyperref für Hyperlink und Sprungtexte
\usepackage{xcolor,hyperref}

\hypersetup{
  pdftitle={Hier den Titel der Arbeit},
  pdfauthor={Hier den Author der Arbeit},
  pdfsubject={Stichworte, weiteres Stichwort},
  colorlinks=true,
  pdfborder={0 0 0},
  bookmarksopen=true,
  bookmarksopenlevel=1,
  bookmarksnumbered=true,
  linkcolor=blue!60!black,
  %linkcolor=black,
  citecolor=blue!60!black,
  urlcolor=blue!60!black,
  filecolor=green!60!black,
  pdfpagemode=UseNone,
  unicode=true,
}

% Paket zum Setzen von Algorithmen in Pseudocode mit kleinen Stilanpassungen
\usepackage[ruled,vlined,linesnumbered,norelsize]{algorithm2e}
\DontPrintSemicolon
\def\NlSty#1{\textnormal{\fontsize{8}{10}\selectfont{}#1}}
\SetKwSty{texttt}
\SetCommentSty{emph}
\def\listalgorithmcfname{Algorithmenverzeichnis}
\def\algorithmautorefname{Algorithmus}
\let\chapter=\section % repariert ein Problem mit algorithm2e

\DeclareOldFontCommand{\bf}{\normalfont\bfseries}{\mathbf}

\usepackage{todonotes}
\usepackage{comment}
\usepackage{algorithm2e}

\newtheorem{definition}{Definition}

\begin{document}

%%%%%%%%%%%%%%%%%%%%%%%%%%%%%%%%%%%%%%%%%%%%%%%%%%%%%%%%%%%%%%%%%%%%%%

\pagestyle{empty} % keine Seitenzahlen

% Titelblatt der Arbeit
\begin{titlepage}

  \begin{center}\large

    \quad\includegraphics[height=17mm]{kit_logo_de.pdf} \hfill
    \includegraphics[height=20mm]{grouplogo-algo-blue.pdf}\quad\null

    \vfill

    Bachelorarbeit
    \vspace*{2cm}

    {\bf\huge Malleable TOHTN Planning \\using CrowdHTN and Mallob \par}
    % Siehe auch oben die Felder pdftitle={}
    % mit \par am Ende stimmt der Zeilenabstand

    \vfill

    Niko Wilhelm

    \vspace*{15mm}

    Abgabedatum: 19.10.2012

    \vspace*{45mm}

    \begin{tabular}{rl}
      Betreuer: & Prof. Dr. Peter Sanders \\
      & M.Sc. Dominik Schreiber \\
    \end{tabular}
    
    \vspace*{10mm}

    Institut für Theoretische Informatik, Algorithmik \\
    Fakultät für Informatik \\
    Karlsruher Institut für Technologie

    % English:
    % Institute of Theoretical Informatics, Algorithmics \\
    % Department of Informatics \\
    % Karlsruhe Institute of Technology

    \vspace*{12mm}
  \end{center}

\end{titlepage}

%%%%%%%%%%%%%%%%%%%%%%%%%%%%%%%%%%%%%%%%%%%%%%%%%%%%%%%%%%%%%%%%%%%%%%

\vspace*{0pt}\vfill

\hrule\medskip

Hiermit versichere ich, dass ich diese Arbeit selbständig verfasst und keine anderen, als die angegebenen Quellen und Hilfsmittel benutzt, die wörtlich oder inhaltlich übernommenen Stellen als solche kenntlich gemacht und die Satzung des Karlsruher Instituts für Technologie zur Sicherung guter wissenschaftlicher Praxis in der jeweils gültigen Fassung beachtet habe.

\bigskip

\noindent
Karlsruhe, den \today

% Unterschrift (handgeschrieben)

\vspace*{5cm}

\clearpage

%%%%%%%%%%%%%%%%%%%%%%%%%%%%%%%%%%%%%%%%%%%%%%%%%%%%%%%%%%%%%%%%%%%%%%

\vspace*{0pt}\vfill

\selectlanguage{german}
\begin{abstract}
\centerline{\bf Zusammenfassung}

Hier die deutsche Zusammenfassung.

Ich bin Blindtext. Von Geburt an. Es hat lange gedauert, bis ich begriffen habe, was es bedeutet, ein blinder Text zu sein: Man macht keinen Sinn. Man wirkt hier und da aus dem Zusammenhang gerissen. Oft wird man gar nicht erst gelesen. Aber bin ich deshalb ein schlechter Text? Ich weiß, dass ich nie die Chance haben werde im Stern zu erscheinen. Aber bin ich darum weniger wichtig? Ich bin blind! Aber ich bin gerne Text. Und sollten Sie mich jetzt tatsächlich zu Ende lesen, dann habe ich etwas geschafft, was den meisten ,,normalen`` Texten nicht gelingt.

Ich bin Blindtext. Von Geburt an. Es hat lange gedauert, bis ich begriffen habe, was es bedeutet, ein blinder Text zu sein: Man macht keinen Sinn. Man wirkt hier und da aus dem Zusammenhang gerissen. Oft wird man gar nicht erst gelesen. Aber bin ich deshalb ein schlechter Text? Ich weiß, dass ich nie die Chance haben werde im Stern zu erscheinen. Aber bin ich darum weniger wichtig? Ich bin blind! Aber ich bin gerne Text.

\end{abstract}

\vfill

\selectlanguage{english}
\begin{abstract}
\centerline{\bf Abstract}

And here an English translation of the German abstract.

I'm blind text. From birth. It took a long time until I realized what it means to be random text: You make no sense. You stand here and there out of context. Frequently, they do not even read. But I have a bad copy? I know that I will never have the chance of appearing in the. But I'm any less important? I'm blind! But I like to text. And you should see me now actually over, then I have accomplished something that is not possible in most ``normal'' copies.

I'm blind text. From birth. It took a long time until I realized what it means to be random text: You make no sense. You stand here and there out of context. Frequently, they do not even read. But I have a bad copy? I know that I will never have the chance of appearing in the. But I'm any less important? I'm blind! But I like to text.

\end{abstract}
\selectlanguage{german}

\vfill\vfill\vfill
\clearpage

%%%%%%%%%%%%%%%%%%%%%%%%%%%%%%%%%%%%%%%%%%%%%%%%%%%%%%%%%%%%%%%%%%%%%%

\vspace*{0pt}\vfill

\section*{Danksagungen}

Thanks to i10pc135 which suffered much to make the experimental evaluation possible.

\vfill\vfill\vfill
\clearpage

%%%%%%%%%%%%%%%%%%%%%%%%%%%%%%%%%%%%%%%%%%%%%%%%%%%%%%%%%%%%%%%%%%%%%%

\pagestyle{normal}
% markiere sections im Seitenkopf links und subsections rechts
\renewcommand\sectionmark[1]{\markboth{\thesection\quad\MakeUppercase{#1}}{\thesection\quad\MakeUppercase{#1}}}
\renewcommand\subsectionmark[1]{\markright{\thesubsection\quad\MakeUppercase{#1}}}

% Inhaltsverzeichnis
\tableofcontents

\clearpage

%%%%%%%%%%%%%%%%%%%%%%%%%%%%%%%%%%%%%%%%%%%%%%%%%%%%%%%%%%%%%%%%%%%%%%

\listoffigures
\listoftables
\listofalgorithms

\clearpage

%%%%%%%%%%%%%%%%%%%%%%%%%%%%%%%%%%%%%%%%%%%%%%%%%%%%%%%%%%%%%%%%%%%%%%

\section{Introduction}
\subsection{Motivation}
\subsection{Research Goal}
- Provide a performant parallel TOHTN planner by improving upon the Crowd planner
- Provide integration of TOHTN into the Mallob malleable load balancer
- Compare performance of parallel to malleable TOHTN planning
- 
\subsection{Thesis Overview}

\section{Theoretical Foundations}

\ref{prelim: tohtn problems}

\subsection{TOHTN Formalism}

\subsubsection{Defining TOHTN Planning Problems}
Both HTN and TOHTN planning are based on the idea of decomposing a list of initial tasks down into smaller subtasks until those subtasks can be achieved by simple actions.

\todo{quote more formalisms}
Multiple definitions for HTN planning exist. In this work we build on the definition introduced in \cite{georgievski2015htn}.

\begin{definition} % predicate
	A \textbf{predicate} consists of two parts. Firstly a predicate symbol $p \in \mathcal{P}$ where $\mathcal{P}$ is the finite set of predicate symbols. Secondly of a list of terms $\tau_1, \ldots, \tau_k$ where each term $\tau_i$ is either a constant symbol $c \in \mathcal{C}$, with $\mathcal{C}$ being the finite set of constant symbols, or a variable symbol $v \in \mathcal{V}$, where $\mathcal{V}$ is the infinite set of variable symbols. \\
	The set of all predicates is called $\mathcal{Q}$.
\end{definition}
With the definition of a predicate in place, we can then define a grounding as well as our world state.
\begin{definition} % grounding
	A \textbf{ground predicate} is a predicate where the terms contain no variable symbols or, in other words, a predicate that contains only constant symbols.
\end{definition}
\begin{definition} % state
	A \textbf{state} $s \in 2^{\mathcal{Q}}$ is a set of ground predicates for which we make the closed-world-assumption. Under the closed-world-assumption, only positive predicates are explicitly represented in $s$. All predicates not in $s$ are implicitly negative.
\end{definition}


\begin{definition} % primitive task/ action
	With $T_p$ the set of primitive task symbols, a \textbf{primitive task} $t_p$ is defined as a triple $t_p(\Tilde{t}_p (a_1, \ldots, a_k), pre(t_p), eff(t_p))$. $\Tilde{p} \in T_p$ is the task symbol, $a_1, \ldots, a_k \in \mathcal{C} \cup \mathcal{V}$ are the task arguments, $pre(t_p) \in 2^{\mathcal{P}}$ the preconditions and $eff(t_p) \in 2^{\mathcal{P}}$ the effects of the primitive task $t_p$. We further define the positive and negative preconditions of $t_p$ as $pre^+(t_p) := \{p \in pre(t_p) : p \text{ is positive}\}$ and $pre^-(t_p) := \{p \in pre(t_p) : p \text{ is negative}\}$. We define $eff^+(t_p)$ and $eff^-(t_p)$ analogously. \\
	We call a fully ground primitive task an \textbf{action}.
\end{definition}

As preconditions and effects may not be concerned with the whole world state the closed-world assumption does not apply to them. To any HTN instance we could create an equivalent one where each precondition and effect cares about the whole world state. This would be achieved by instantiating all the "don't care" terms in preconditions and effects with all possible combinations of predicates. Doing this would, however, come at the price of a huge blowup of our planning problem. \\

\begin{definition} % applicable, application
	An action $t_p$ is \textbf{applicable}
	- $pre^+$ is fully contained in the state $s$
	- none of $pre^-$ are contained in $s$
	- the \textbf{application} of a task gives us a new state $s' = (s \setminus eff^-(t_p)) \cup eff^+(t_p)$
\end{definition}

\begin{definition} % compound task/ abstract task
	We define a \textbf{compound task} as $t_c = \Tilde{t}_c(a_1, \ldots, a_k)$, where $\Tilde{t_c} \in T_c$ is the task symbol from the finite set of compound task symbols $T_c$ and $a_1, \ldots, a_k$ are the task arguments.
\end{definition}
Primitive and compound tasks together form task networks. In places where both can be used, we will refer to them simply as tasks $t \in T$.

\begin{definition} % task network
	Let $T = T_p \bigcup T_c$ be a set of primitive and compound tasks. A task network is a tuple $\tau = (T, \psi)$ consisting of tasks $T$ and constraints $\psi$ between those tasks.
\end{definition}

\begin{definition} % method, reduction
	Let $M$ be a finite set of method symbols and $T = T_p \bigcup T_c$ a set of primitive and compound tasks. A \textbf{method} $m = (\Tilde{m}(a_1, \ldots, a_k), t_c, pre(m), subtasks(m), constraints(m))$ is a tuple consisting of the method symbol $\Tilde{m}$, the method arguments $a_1, \ldots, a_k$, the associated compound task $t_c \in T_c$ the method refers to, a set of preconditions $pre(m) \in 2^{\mathcal{P}}$, a set of tasks $subtasks(m) = \{t_1, \ldots, t_l\}, t_i \in T$ and a set of constraints $c_1, \ldots, c_m$ defining relationships between the tasks in $subtasks(m)$. Any arguments appearing in $t_c, pre(m), subtasks(m)$ must also appear in $a_1, \ldots, a_k$.\\
	In TOHTN planning, $constraints(m)$ is implicitly set s.t. the subtasks $t_1, \ldots, t_l$ are totally ordered. \\
	We call a fully ground method a \textbf{reduction}.
\end{definition}
\todo{link directly to resolution of a task network if resolving is defined before method!}
Each method $m$ has exactly one associated compound task $t_c$. However, multiple methods $m_1, \ldots, m_k$ may be associated with a single compound task $t_c$. Additionally, while any arguments of $t_c$ must be present in $m$, the contrary is not true and $m$ may have arguments not present in $t_c$, i.e., $m$ is not fully determined by $t_c$. As a result methods present choice points both in the choice of method itself as well as through the argument instantiation. \\


\begin{definition}
	Test
\end{definition}

\

\subsection{Techniques to solve TOHTN planning problems}
- SAT-based
- search-based
- lifted vs grounded

\subsection{Malleable Algorithms}

\section{TOHTN Metadata}
- researchers in TOHTN planning have collected a number of test instances for TOHTN planning
- provide some analysis of those instances in the context of modelling TOHTN planning as graph search
- provide a foundation to discuss the effects of changes and improvements in the crowd planning framework

\section{Implementation}

\include{crowd_improvements}

\section{Theoretical Improvements to the Crowd Planner}
\section{Search Algorithms Used in CrowdHTN}

\subsection{Random DFS}

\subsection{Random BFS}
- trivially complete, will always encounter everything
- reference statistics (branching factor from Crowd, minimal plan depth from Lilotane) to show that 

\subsection{GBFS}
- inspired by PandaGBFS
- GBFS:
	- use a heuristic to grade each node
	- next node: best of all neighbors of the current node
	- then do a DFS on this
- GBFS is not complete
- might get stuck in an infinite loop
\todo{Quote to describe GBFS?}

The heuristic:
- determine the minimum number of actions/ reductions needed to solve the task network
- ignore any and all parameters and world state to simplify the computation
\todo{ensure the code actually does this!}
- iterative approach:
	- initial:
		- actions have remaining depth 0
		- reductions without children/ where all children are noops have depth 1
	- iterating:
		- reductions: the sum of the minimum depths of all subtasks + 1
		- tasks: the minimum over all reductions
- i.e., we heuristically try to find nodes with the minimum amount of work remaining

\begin{algorithm}
	\caption{GBFS heuristic calculation}
	task depths $\gets \{(t_c, 0) | \text{t_c is concrete task}\}$\;
	\While{task depths changed}{
		
	}
\end{algorithm}


\subsection{A-Star}
- same heuristic as GBFS
- also track the number of applied reductions so far
- gives us completeness (even without loop detection!) as the length of the path so far gives us a kind of BFS characteristic to the whole thing
- for each node we know exactly that any path
- we modify A* to terminate as soon as we find a goal for the first time
- could turn into an anytime algorithm by keeping up the search until optimal plan


\include{loop_detection}

\section{Malleable TOHTN Planning}
- ensure completeness
- ensure performance
- work stealing is kinda nice here

\include{grounding_pruning}

\section{Experimental Evaluation}

\section{Future Work}

\clearpage

%%%%%%%%%%%%%%%%%%%%%%%%%%%%%%%%%%%%%%%%%%%%%%%%%%%%%%%%%%%%%%%%%%%%%%

\bibliographystyle{gerplain}
\bibliography{literatur}

\end{document}
