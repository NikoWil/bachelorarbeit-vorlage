\subsection{Techniques to solve TOHTN planning problems}

\subsection{SAT-based}
- SAT-based
- often with a BFS-like characteristic (Tree-Rex, Lilotane, TotSat)

\subsubsection{Search-based}
- take definitions from \cite{holler2020htn} unless noted otherwise
- explain where we differ from HTN progression search (holler2020htn also does this)
- 
The definitions of plan space search and progression search in this section are taken from \cite{holler2020htn} unless noted otherwise.

\paragraph{Plan Space Search}
- search for parts of plans
- 

\paragraph{Progression Search}
\cite{holler2020htn}
	- progression search is one of the best known search algorithms
	- generate plans in a forward way
	- always resolve a task that has no more open predecessors with the ordering constraints (is called 'unconstrained task')
		- own: for TOHTN: we always have exactly 1 task we want to process next!
		- makes it trivial to find the next unconstrained task
		- mentioned in the paper
	- progression search planners always know the current (world) state, can use this information for heuristics, pruning etc
	
	- other planners search partial plans but not in order, they thus cannot know the current world state
	- perform goal test on popping: find optimal plan if popping order is informed by cost
	- perform goal test before popping: explore fewer nodes
	
	- Alford et al, 2012, Thm. 3 -> HTN problem is solvable <=> there is a solution in progression space
	
	- parts of the search space will be searched more than once if no additional measures are taken
	
\begin{algorithm}
	\caption{Classical Progression Search for HTN as introduced in \cite{holler2020htn}}
	$fringe \gets \{ (s_0, tn_I, \epsilon)\}$\;
	\While{$fringe \neq \emptyset$}{
		$n \gets fringe.pop()$\;
		\If{$n.isgoal$}{
			\textbf{return} $n$\;
		}
		$U \gets n.unconstrainedNodes$\;
		\For{$t \in U$}{
			\eIf{$isPrimitive(t)$}{
				\If{$isApplicable(t)$}{
					$n' \gets n.apply(t)$\;
					$fringe.add(n')$\;
				}
			}{
				\For{$m \in t.methods$}{
					$n' \gets n.decompose(t, m)$\;
					$fringe.add(n')$\;
				}
			}
		}
	}
\end{algorithm}

- search-based
- lifted vs grounded
	- Lilotane, HyperTensioN
	- Panda, CrowdHTN, Tree-Rex
	- lifted: more general, less pruning?